\documentclass[12pt,letterpaper]{article}

% ----------------------Preamble---------------------------------------------
\usepackage[utf8]{inputenc}
\usepackage[margin=1in]{geometry}
\usepackage[style=ieee]{biblatex}
\usepackage{graphicx}
\addbibresource{references.bib}
\usepackage{gensymb}

%% Page Setup
\setlength{\parskip}{1em}
\setlength{\parindent}{0em}
% ---------------------------------------------------------------------------

\begin{document}
    % -------------------------Title-----------------------------------------
    \begin{center}
        \begin{figure}
            \centering
            \includegraphics[width=0.125\linewidth]{nakujalogo.jpeg}
        \end{figure}
        \huge{Safety Manual}             \\
        \rule{\textwidth}{0em}
        \large{on} 							\\
        \rule{\textwidth}{0em}
        \large{Wednesday, 13 February, 2025}
    \end{center}
    % -----------------------------------------------------------------------

    % -------------------------Introduction----------------------------------
    \section*{Introduction}
        \noindent The static test is an exercise that entails the firing a
        rocket motor mounted in a stationary structure for the purpose of
        obtaining information on the motor’s performance. The motor is a
        high power device that poses risks to the personnel involved with
        the test, therefore it is necessary to create a safety manual in
        order to address potential hazards and procedures of mitigating
        them as well as protocols to be followed in case of any accidents
        or abnormalities. 
    % ----------------------------------------------------------------------

    % --------------------------Body----------------------------------------
    \section{Static Test}
        \subsection{Locating the Test Site}
            \noindent A suitable location is an area of level ground and
            open space. The area should be clear of buildings or other
            structures. It shouldn’t be an area that is frequented by
            humans and animals.
        \subsection{Personnel Control}
            \noindent Success of the test area safety is highly dependent
            on proper conduct of personnel in the test environment. Below
            are procedures that are regarded as minimum requirements for
            safe control of personnel within the test area.
            \begin{enumerate}
                \begin{enumerate}                  
                    \item An observer is required to visit the site to
                    confirm the worthiness of the site.
                    \item An appropriate location of placing the setup
                    is chosen based on the minimum personnel distance.
                    \item Non-involved parties around the test area
                    should be warned of the test and marshaled to or
                    further away from the minimum safe distance. 
                    \item If possible a warning sign can be placed on
                    the site entrance informing people of the test 
                    \item Only authorized personnel should be allowed to
                    handle potentially hazardous materials during
                    transportation and assembly.
                    \item Before commencing the set up process, any
                    personnel smoking should be directed to a designated
                    smoking area.
                    \item The personnel setting up the test should be in
                    proper gear for the tasks.
                    \item Before firing the motor, a suitable audio warning
                    system should be activated a few minutes into the start
                    of the test
                \end{enumerate}
            \end{enumerate}
        \subsection{Preparation of the Setup}
            \begin{enumerate}
                 \begin{enumerate}
                     \item The components should be inspected for any lose
                     connections and mended. 
                     \item The components should be cleaned and rid of any
                     unnecessary objects before being transported to the test
                     site. 
                     \item A checklist should be used for every assembly of
                     components in the setup. It is recommended that the
                     checklist be used during and after assembly of the
                     components. 
                     \item After the test unit is restrained, the strength
                     of the restraint should be checked by a couple of
                     personnel and approved for sound strength. 
                     \item It is at this point that the connection of the
                     electrical firing circuit is done to the test unit
                     from the bay. All personnel except arming crew must be
                     evacuated from the test bay and area will be cleared
                     before arming procedures are started.
                 \end{enumerate}
            \end{enumerate}
        \subsection{Firing the Motor}
            \begin{enumerate}
                \begin{enumerate}
                    \item Countdown will start after everyone is within the
                    approved safe areas. It should consist of verbal communication.
                    In distant areas from the test unit, a public address system
                    should be used. 
                    \item The motor is fired on the command “Ignition!”
                \end{enumerate}
            \end{enumerate}
    \section{Abnormal Test Conditions}
        \subsection{Missfire}
            \noindent A misfire is defined as any failure to ignite the test
            unit. After several ignition attempts fail, below are the steps to
            be followed.
            \begin{enumerate}
                \begin{enumerate}
                    \item In the event of a misfire, a wait will be required as
                    determined by the test conductor in case of a hanging fire
                    (ideally 2-5 minutes).
                    \item After the approved waiting period, the ignition circuit
                    is disarmed and only qualified personnel will proceed to the
                    test unit site for inspection and fault diagnosis.
                    \item Only after the setup is deemed safe personnel in the
                    site is when other additional parties will be allowed to
                    access the test site.
                \end{enumerate}
            \end{enumerate}
        \subsection{Hang Fire}
            \noindent A hang fire is defined as a firing with undue ignition
            delay. The hang fire will be treated as a normal firing after the
            motor has been fired.
    \section{Malfunction}
        \noindent A malfunction involves case separation, nozzle \&
        bulkhead failure and detonation.
            \begin{enumerate}
                \begin{enumerate}
                    \item In the event of a malfunction, the area will be
                    checked for fires by the test personnel and appropriate
                    action will be taken to suppress the fire. In case of a
                    large fire, the nearest fire alarm should be triggered.
                    \item A wait will be required as determined by the test
                    personnel in order to assess the safety of accessing the
                    site where the malfunction has occurred.
                    \item After the approved waiting period, an inspection
                    team will proceed to the test site to investigate the
                    hazardous conditions.
                    \item Hazardous debri will be cleared before the
                    “all clear” is sounded.
                    \item A thorough inspection for any unburned propellant
                    should be done and isolated from any hot objects.\cite{dod1962safety}
                \end{enumerate}
            \end{enumerate}
    \section{Electrical Safety}
        \noindent To be Completed
    \section{Chemical Safety}
        \subsection{Storage of Chemicals}
            \begin{enumerate}
                \begin{enumerate}
                \item All chemicals should be stored in their proper
                container that is well-sealed and clearly labeled in a
                dry, well-aerated cabinet or room.
                \item Chemicals should be stored by hazard class i.e
                flammable liquids, organic substances, oxidisers.\footnotemark[1]\footnotemark[2]
                \footnotetext[1]{Flammable liquids are mostly Organic
                substances like alcohols and ketones. Oxidisers could be
                acids and some salts}
                \footnotetext[2]{Never store a ketone with hydrogen peroxide
                in not well sealed containers. Trust me.}
                \item Ensure you follow the manufacturer's specifications
                on storage of the chemical.\footnotemark[3]
                \footnotetext[3]{It's usually at indicated on the container.
                If not, then that is a poor product.}
                \end{enumerate}
            \end{enumerate}
        \subsection{Safe handling of Chemicals}
            \begin{enumerate}
                \begin{enumerate}
                    \item Don't ingest, inhale or taste directly any chemicals.\footnotemark[4]
                    \footnotetext[4]{Avoid having consumables in the Prototyping lab to reduce the risk
                    of contamination}
                    \item Prefer using gloves when working with chemicals, as well as wearing closed
                    shoes and overalls.\footnotemark[5]
                    \footnotetext[5]{Remember that hearing and noise protection, as well as goggles is advised
                    while working with volatile compounds}
                    \item Keep your working area clean and tidy before, during and after working with chemicals.\footnotemark[6]
                    \footnotetext[6]{This will help you have room to manoeuvre the
                    apparatus and reagents you're currently working with. You'll
                    prevent chances of spillage and damaging equipment, also.}
                \end{enumerate}
            \end{enumerate}
    	\subsection{Waste Disposal}
      		\begin{enumerate}
                    \begin{enumerate}
    				\item All chemical waste should be conveyed to the Chemical 
                        and/or Engineering Labs for safe disposal.\cite{nema2006disposal}
                        \item Ensure that waste is separated in terms of  ignitablility, corrosivity, reactivity and toxicity, in addition to being well
                        labeled.\cite{acs2007manual}
                    \end{enumerate}
                \end{enumerate}
    \pagebreak
    % -----------------------------------------------------------------------

    % --------------------------Bibliography---------------------------------
    \printbibliography
    % -----------------------------------------------------------------------
\end{document}
